\documentclass[titlepage]{article}

\usepackage{amsmath}
\usepackage{biblatex}
\usepackage{chemformula}
\usepackage{fancyvrb}
\usepackage{graphicx}
\usepackage{hyperref}
\usepackage{lipsum}
\usepackage[margin=1in]{geometry}
\usepackage{xcolor}

\addbibresource{TS2_refs.bib}

\DefineVerbatimEnvironment{blockcode}
  {Verbatim}
  {fontsize=\small,formatcom=\color{blue}}

\begin{document}

\title{Custom Rate Constants - Beyond TS1 and TS2}
\author{Matt Dawson}
\maketitle

%%%%%%%%%%%%%%%%%%%%%%%%%%%%%%%%%%%%%%%%%%%%%%%

\section{Using this Document}

Contained are notes on the coversion of all custom rate constant equations
from CAM to standard MICM reaction types, building on
the custom rate constant equation notes for TS1 and TS2. The original
rate constant functions are taken from the CAM source code, at 
\verb>cam/src/chemistry/mozart/mo_usrrxt.F90>. These notes represent
a snapshot in time, and the CAM code will likely have changed since this
document was written. The code snippets are included to provide context. 


%%%%%%%%%%%%%%%%%%%%%%%%%%%%%%%%%%%%%%%%%%%%%%%

\section{usr\_MCO3\_XNO2}

\begin{blockcode}[commandchars=\\\{\}]
\color{gray}/cam/src/chemistry/mozart/mo_usrrxt.F90 (796,796)
       tp(:)             = 300._r8 * tinv(:)
\end{blockcode}       
\begin{blockcode}[commandchars=\\\{\}]
\color{gray}/cam/src/chemistry/mozart/mo_usrrxt.F90 (958,966)
!-----------------------------------------------------------------
!     ... mco3 + no2 -> mpan
!-----------------------------------------------------------------
       if( usr_MCO3_NO2_ndx > 0 ) then
          rxt(:,k,usr_MCO3_NO2_ndx) = 1.1e-11_r8 * tp(:) / m(:,k)
       end if
       if( usr_MCO3_XNO2_ndx > 0 ) then
          rxt(:,k,usr_MCO3_XNO2_ndx) = 1.1e-11_r8 * tp(:) / m(:,k)
       end if
\end{blockcode}

As discussed in the TS1 document sections 11 and 19, this appears to have been replaced
by a Troe reaction. The proposed reworking of this function (TS1 section 19) included
removing M from the reactants and products and using a standard Arrhenius reaction:

\begin{equation}
k = Ae^{(\frac{C}{T})}(\frac{T}{D})^B(1.0+E \times P),
\end{equation}

\noindent with $A = 1.1 \times 10^{-11}$, $B = -1$, $C = 0$, $D = 300$, and $E = 0$.

However, it appears that the CAM preprocessor may ignore M as a reactant, meaning that
this reaction would require a new reaction type in Music Box.

\vspace{20px}
\textit{\Large Should a new reaction type be added to Music Box for this reaction? Or is
this no longer used?}

%%%%%%%%%%%%%%%%%%%%%%%%%%%%%%%%%%%%%%%%%%%%%%%

\section{usr\_XOOH\_OH}

\begin{blockcode}[commandchars=\\\{\}]
\color{gray}/cam/src/chemistry/mozart/mo_usrrxt.F90 (1046,1052)
!-----------------------------------------------------------------
!       ... xooh + oh -> h2o + oh
!-----------------------------------------------------------------
       if( usr_XOOH_OH_ndx > 0 ) then
          call comp_exp( exp_fac, 253._r8*tinv, ncol )
          rxt(:,k,usr_XOOH_OH_ndx) = temp(:ncol,k)**2._r8 * 7.69e-17_r8 * exp_fac(:)
       end if
\end{blockcode}

This can be rearranged as an Arrhenius reaction:

\begin{equation}
k = Ae^{(\frac{C}{T})}(\frac{T}{D})^B(1.0+E \times P),
\end{equation}

\noindent with $A = 7.69 \times 10^{-17}$, $B = 2$, $C = 253$, $D = 1$, and $E = 0$.


%%%%%%%%%%%%%%%%%%%%%%%%%%%%%%%%%%%%%%%%%%%%%%%

\section{usr\_OA\_O2}

\begin{blockcode}[commandchars=\\\{\}]
\color{gray}/cam/src/chemistry/mozart/mo_usrrxt.F90 (800,808)
!-----------------------------------------------------------------
! ... o + o2 + m --> o3 + m (JPL15-10)
!-----------------------------------------------------------------
       if( usr_O_O2_ndx > 0 ) then
          rxt(:,k,usr_O_O2_ndx) = 6.e-34_r8 * tp(:)**2.4_r8
       end if
       if( usr_OA_O2_ndx > 0 ) then
          rxt(:,k,usr_OA_O2_ndx) = 6.e-34_r8 * tp(:)**2.4_r8
       end if
\end{blockcode}

Same as \verb>usr_O_O2> (TS1 section 5).

%%%%%%%%%%%%%%%%%%%%%%%%%%%%%%%%%%%%%%%%%%%%%%%

\section{usr\_XNO2NO3\_M and usr\_NO2XNO3\_M}

\begin{blockcode}[commandchars=\\\{\}]
\color{gray}/cam/src/chemistry/mozart/mo_usrrxt.F90 (849,875)
!-----------------------------------------------------------------
! ... n2o5 + m --> no2 + no3 + m (JPL15-10)
!-----------------------------------------------------------------
       if( usr_N2O5_M_ndx > 0 ) then
          if( tag_NO2_NO3_ndx > 0 ) then
             call comp_exp( exp_fac, -10840.0_r8*tinv, ncol )
             rxt(:,k,usr_N2O5_M_ndx) = rxt(:,k,tag_NO2_NO3_ndx) * 1.724138e26_r8 * exp_fac(:)
          else
             rxt(:,k,usr_N2O5_M_ndx) = 0._r8
          end if
       end if
       if( usr_XNO2NO3_M_ndx > 0 ) then
          if( tag_NO2_NO3_ndx > 0 ) then
             call comp_exp( exp_fac, -10840.0_r8*tinv, ncol )
             rxt(:,k,usr_XNO2NO3_M_ndx) = rxt(:,k,tag_NO2_NO3_ndx) *1.724138e26_r8 * exp_fac(:)
          else
             rxt(:,k,usr_XNO2NO3_M_ndx) = 0._r8
          end if
       end if
       if( usr_NO2XNO3_M_ndx > 0 ) then
          if( tag_NO2_NO3_ndx > 0 ) then
             call comp_exp( exp_fac, -10840.0_r8*tinv, ncol )
             rxt(:,k,usr_NO2XNO3_M_ndx) = rxt(:,k,tag_NO2_NO3_ndx) * 1.734138e26_r8 * exp_fac(:)
          else
             rxt(:,k,usr_NO2XNO3_M_ndx) = 0._r8
          end if
       end if
\end{blockcode}
 
Same as \verb>usr_N2O5_M> (TS1 section 7)
       
%%%%%%%%%%%%%%%%%%%%%%%%%%%%%%%%%%%%%%%%%%%%%%%

\section{usr\_XHNO3\_OH}

\begin{blockcode}[commandchars=\\\{\}]
\color{gray}/cam/src/chemistry/mozart/mo_usrrxt.F90 (877,897)
!-----------------------------------------------------------------
! set rates for:
!   ... hno3 + oh --> no3 + h2o
!           ho2no2 + m --> ho2 + no2 + m
!-----------------------------------------------------------------
       if( usr_HNO3_OH_ndx > 0 ) then
          call comp_exp( exp_fac, 1335._r8*tinv, ncol )
          ko(:) = m(:,k) * 6.5e-34_r8 * exp_fac(:)
          call comp_exp( exp_fac, 2199._r8*tinv, ncol )
          ko(:) = ko(:) / (1._r8 + ko(:)/(2.7e-17_r8*exp_fac(:)))
          call comp_exp( exp_fac, 460._r8*tinv, ncol )
          rxt(:,k,usr_HNO3_OH_ndx) = ko(:) + 2.4e-14_r8*exp_fac(:)
       end if
       if( usr_XHNO3_OH_ndx > 0 ) then
          call comp_exp( exp_fac, 1335._r8*tinv, ncol )
          ko(:) = m(:,k) * 6.5e-34_r8 * exp_fac(:)
          call comp_exp( exp_fac, 2199._r8*tinv, ncol )
          ko(:) = ko(:) / (1._r8 + ko(:)/(2.7e-17_r8*exp_fac(:)))
          call comp_exp( exp_fac, 460._r8*tinv, ncol )
          rxt(:,k,usr_XHNO3_OH_ndx) = ko(:) + 2.4e-14_r8*exp_fac(:)
       end if
\end{blockcode}      

Same as \verb>usr_HNO3_OH> (TS1 section 18).

%%%%%%%%%%%%%%%%%%%%%%%%%%%%%%%%%%%%%%%%%%%%%%%

\section{usr\_XHO2NO2\_M}

\begin{blockcode}[commandchars=\\\{\}]
\color{gray}/cam/src/chemistry/mozart/mo_usrrxt.F90 (898,913)
       if( usr_HO2NO2_M_ndx > 0 ) then
          if( tag_NO2_HO2_ndx > 0 ) then
             call comp_exp( exp_fac, -10900._r8*tinv, ncol )
             rxt(:,k,usr_HO2NO2_M_ndx) = rxt(:,k,tag_NO2_HO2_ndx) * exp_fac(:) / 2.1e-27_r8
          else
             rxt(:,k,usr_HO2NO2_M_ndx) = 0._r8
          end if
       end if
       if( usr_XHO2NO2_M_ndx > 0 ) then
          if( tag_NO2_HO2_ndx > 0 ) then
             call comp_exp( exp_fac, -10900._r8*tinv, ncol )
             rxt(:,k,usr_XHO2NO2_M_ndx) = rxt(:,k,tag_NO2_HO2_ndx) * exp_fac(:) / 2.1e-27_r8
          else
             rxt(:,k,usr_XHO2NO2_M_ndx) = 0._r8
          end if
       end if
\end{blockcode}      

Same as \verb>usr_HO2NO2_M> (TS1 section 8).

%%%%%%%%%%%%%%%%%%%%%%%%%%%%%%%%%%%%%%%%%%%%%%%

\section{usr\_XPAN\_M}

\begin{blockcode}[commandchars=\\\{\}]
\color{gray}/cam/src/chemistry/mozart/mo_usrrxt.F90 (968,985)
!-----------------------------------------------------------------
! ... pan + m --> ch3co3 + no2 + m (JPL15-10)
!-----------------------------------------------------------------
       call comp_exp( exp_fac, -14000._r8*tinv, ncol )
       if( usr_PAN_M_ndx > 0 ) then
          if( tag_CH3CO3_NO2_ndx > 0 ) then
             rxt(:,k,usr_PAN_M_ndx) = rxt(:,k,tag_CH3CO3_NO2_ndx) * 1.111e28_r8 * exp_fac(:)
          else
             rxt(:,k,usr_PAN_M_ndx) = 0._r8
          end if
       end if
       if( usr_XPAN_M_ndx > 0 ) then
          if( tag_CH3CO3_NO2_ndx > 0 ) then
             rxt(:,k,usr_XPAN_M_ndx) = rxt(:,k,tag_CH3CO3_NO2_ndx) * 1.111e28_r8 * exp_fac(:)
          else
             rxt(:,k,usr_XPAN_M_ndx) = 0._r8
          end if
       end if
\end{blockcode}      

Same as \verb>usr_PAN_M> (TS1 section 17).

%%%%%%%%%%%%%%%%%%%%%%%%%%%%%%%%%%%%%%%%%%%%%%%

\section{usr\_XMPAN\_M}

\begin{blockcode}[commandchars=\\\{\}]
\color{gray}/cam/src/chemistry/mozart/mo_usrrxt.F90 (987,1003)
!-----------------------------------------------------------------
! ... mpan + m --> mco3 + no2 + m (JPL15-10)
!-----------------------------------------------------------------
       if( usr_MPAN_M_ndx > 0 ) then
          if( tag_MCO3_NO2_ndx > 0 ) then
             rxt(:,k,usr_MPAN_M_ndx) = rxt(:,k,tag_MCO3_NO2_ndx) * 1.111e28_r8 * exp_fac(:)
          else
             rxt(:,k,usr_MPAN_M_ndx) = 0._r8
          end if
       end if
       if( usr_XMPAN_M_ndx > 0 ) then
          if( tag_MCO3_NO2_ndx > 0 ) then
             rxt(:,k,usr_XMPAN_M_ndx) = rxt(:,k,tag_MCO3_NO2_ndx) * 1.111e28_r8 * exp_fac(:)
          else
             rxt(:,k,usr_XMPAN_M_ndx) = 0._r8
          end if
       end if
\end{blockcode}      

Same as \verb>usr_MPAN_M> (TS1 section 11).

%%%%%%%%%%%%%%%%%%%%%%%%%%%%%%%%%%%%%%%%%%%%%%%

\section{usr\_C2O3\_NO2}

\begin{blockcode}[commandchars=\\\{\}]
\color{gray}/cam/src/chemistry/mozart/mo_usrrxt.F90 (1444,1449)
       if ( usr_C2O3_NO2_ndx > 0 ) then
          ko(:)   = 2.6e-28_r8 * m(:,k)
          kinf(:) = 1.2e-11_r8
          rxt(:,k,usr_C2O3_NO2_ndx) = (ko/(1._r8+ko/kinf)) * 0.6_r8**(1._r8/(1._r8+(log10(ko/kinf))**2))
       end if
\end{blockcode}  

This is a Troe reaction:

\begin{equation}
\begin{split}
k & = \frac{k_0[\mbox{M}]}{1+k_0[\mbox{M}]/k_{\inf}}F_C^{(1+1/N[log_{10}(k_0[\mbox{M}]/k_{\inf})]^2)^{-1}} \\
k_0 & = A_0 e^{\left( \frac{C_0}{T} \right)} \left( \frac{T}{300} \right)^{B_0} \\
k_{inf} & = A_{inf} e^{\left( \frac{C_{inf}}{T} \right)} \left( \frac{T}{300} \right)^{B_{inf}}
\end{split}
\end{equation}

\noindent where $F_C = 0.6$, $N = 1$, $A_0 = 2.6 \times 10^{-28}$, $B_0 = 0$, $C_0 = 0$, $A_{inf} = 1.2 \times 10^{-11}$, $B_{inf} = 0$, and $C_{inf} = 0$.


%%%%%%%%%%%%%%%%%%%%%%%%%%%%%%%%%%%%%%%%%%%%%%%

\section{usr\_C2H4\_OH}

\begin{blockcode}[commandchars=\\\{\}]
\color{gray}/cam/src/chemistry/mozart/mo_usrrxt.F90 (1454,1458)
       if ( usr_C2H4_OH_ndx > 0 ) then
          ko(:)   = 1.0e-28_r8 * m(:,k)
          kinf(:) = 8.8e-12_r8
          rxt(:,k,usr_C2H4_OH_ndx) = (ko/(1._r8+ko/kinf)) * 0.6_r8**(1._r8/(1._r8+(log(ko/kinf))**2))
       end if
\end{blockcode}  

This is a Troe reaction:

\begin{equation}
\begin{split}
k & = \frac{k_0[\mbox{M}]}{1+k_0[\mbox{M}]/k_{\inf}}F_C^{(1+1/N[log_{10}(k_0[\mbox{M}]/k_{\inf})]^2)^{-1}} \\
k_0 & = A_0 e^{\left( \frac{C_0}{T} \right)} \left( \frac{T}{300} \right)^{B_0} \\
k_{inf} & = A_{inf} e^{\left( \frac{C_{inf}}{T} \right)} \left( \frac{T}{300} \right)^{B_{inf}}
\end{split}
\end{equation}

\noindent where $F_C = 0.6$, $N = 1$, $A_0 = 1.0 \times 10^{-28}$, $B_0 = 0$, $C_0 = 0$, $A_{inf} = 8.8 \times 10^{-12}$, $B_{inf} = 0$, and $C_{inf} = 0$.


%%%%%%%%%%%%%%%%%%%%%%%%%%%%%%%%%%%%%%%%%%%%%%%

\section{usr\_XO2N\_HO2}

\begin{blockcode}[commandchars=\\\{\}]
\color{gray}/cam/src/chemistry/mozart/mo_usrrxt.F90 (1459,1461)
       if ( usr_XO2N_HO2_ndx > 0 ) then
          rxt(:,k,usr_XO2N_HO2_ndx) = rxt(:,k,tag_XO2N_NO_ndx)* \mbox{\color{black}(continued)}
            rxt(:,k,tag_XO2_HO2_ndx)/(rxt(:,k,tag_XO2_NO_ndx)+1.e-36_r8)
       end if
\end{blockcode}  

Looking through the Chemistry Caf\'{e}, the \verb>tag_XO2_HO2_ndx> appears to be an Arrhenius
reaction with $A = 8 \times 10^{-13}$ and $C = 700$, and the \verb>tag_XO2_NO_ndx> appears to
also be an Arrhenius reaction with $A = 2.7 \times 10^{-12}$ and $C = 360$, but I can't find
anything corresponding to \verb>tag_XO2N_NO_ndx> in the database or the mechanisms that are
stored with the CAM source code.

\vspace{20px}
\textit{\Large How should this rate constant be calculated?}

%%%%%%%%%%%%%%%%%%%%%%%%%%%%%%%%%%%%%%%%%%%%%%%

\section{usr\_C2O3\_XNO2}

\begin{blockcode}[commandchars=\\\{\}]
\color{gray}/cam/src/chemistry/mozart/mo_usrrxt.F90 (1449,1453)
       if ( usr_C2O3_XNO2_ndx > 0 ) then
          ko(:)   = 2.6e-28_r8 * m(:,k)
          kinf(:) = 1.2e-11_r8
          rxt(:,k,usr_C2O3_XNO2_ndx) = (ko/(1._r8+ko/kinf)) * 0.6_r8**(1._r8/(1._r8+(log10(ko/kinf))**2))
       end if
\end{blockcode} 

This is a Troe reaction:

\begin{equation}
\begin{split}
k & = \frac{k_0[\mbox{M}]}{1+k_0[\mbox{M}]/k_{\inf}}F_C^{(1+1/N[log_{10}(k_0[\mbox{M}]/k_{\inf})]^2)^{-1}} \\
k_0 & = A_0 e^{\left( \frac{C_0}{T} \right)} \left( \frac{T}{300} \right)^{B_0} \\
k_{inf} & = A_{inf} e^{\left( \frac{C_{inf}}{T} \right)} \left( \frac{T}{300} \right)^{B_{inf}}
\end{split}
\end{equation}

\noindent where $F_C = 0.6$, $N = 1$, $A_0 = 2.6 \times 10^{-28}$, $B_0 = 0$, $C_0 = 0$, $A_{inf} = 1.2 \times 10^{-11}$, $B_{inf} = 0$, and $C_{inf} = 0$.

%%%%%%%%%%%%%%%%%%%%%%%%%%%%%%%%%%%%%%%%%%%%%%%

\section{usr\_CLm\_H2O\_M}

\begin{blockcode}[commandchars=\\\{\}]
\color{gray}/cam/src/chemistry/mozart/mo_usrrxt.F90 (1909,1910)
        call comp_exp( exp_fac, -6600._r8 * tinv, ncol )
        rxt(:,k,usr_clm_h2o_m_ndx) = 2.e-8_r8 * exp_fac(:)
\end{blockcode} 

This is an Arrhenius reaction:

\begin{equation}
k = Ae^{(\frac{C}{T})}(\frac{T}{D})^B(1.0+E \times P),
\end{equation}

\noindent with $A = 2 \times 10^{-8}$, $B = 0$, $C = -6600$, and $E = 0$.

%%%%%%%%%%%%%%%%%%%%%%%%%%%%%%%%%%%%%%%%%%%%%%%

\section{usr\_CLm\_HCL\_M}

\begin{blockcode}[commandchars=\\\{\}]
\color{gray}/cam/src/chemistry/mozart/mo_usrrxt.F90 (1912,1913)
        call comp_exp( exp_fac, -11926._r8 * tinv, ncol )
        rxt(:,k,usr_clm_hcl_m_ndx) =  tinv(:) * exp_fac(:)
\end{blockcode} 

This is an Arrhenius reaction:

\begin{equation}
k = Ae^{(\frac{C}{T})}(\frac{T}{D})^B(1.0+E \times P),
\end{equation}

\noindent with $A = 1$, $B = -1$, $C = -11926$, $D = 1$, and $E = 0$.

%%%%%%%%%%%%%%%%%%%%%%%%%%%%%%%%%%%%%%%%%%%%%%%

\section{usr\_oh\_co}

\begin{blockcode}[commandchars=\\\{\}]
\color{gray}/cam/src/chemistry/mozart/mo_usrrxt.F90 (796,796)
       tp(:)             = 300._r8 * tinv(:)
\end{blockcode}
\begin{blockcode}[commandchars=\\\{\}]
\color{gray}/cam/src/chemistry/mozart/mo_usrrxt.F90 (1757,1770)
       !-----------------------------------------------------------------------
       !     ... CO + OH --> CO2 + HO2
       !-----------------------------------------------------------------------
       if ( usr_oh_co_ndx > 0 ) then
          ko(:)     = 5.9e-33_r8 * tp(:)**1.4_r8
          kinf(:)   = 1.1e-12_r8 * (temp(:ncol,k) / 300._r8)**1.3_r8
          ko_m(:)   = ko(:) * m(:,k)
          k0(:)     = 1.5e-13_r8 * (temp(:ncol,k) / 300._r8)**0.6_r8
          kinf_m(:) = (2.1e+09_r8 * (temp(:ncol,k) / 300._r8)**6.1_r8) / m(:,k)
          rxt(:,k,usr_oh_co_ndx) = (ko_m(:)/(1._r8+(ko_m(:)/kinf(:)))) * &
               0.6_r8**(1._r8/(1._r8+(log10(ko_m(:)/kinf(:)))**2._r8)) + &
               (k0(:)/(1._r8+(k0(:)/kinf_m(:)))) * &
               0.6_r8**(1._r8/(1._r8+(log10(k0(:)/kinf_m(:)))**2._r8))
       endif
\end{blockcode}

This rate constant is the sum of a Troe rate constant:

\begin{equation}
\begin{split}
k & = \frac{k_0[\mbox{M}]}{1+k_0[\mbox{M}]/k_{\inf}}F_C^{(1+1/N[log_{10}(k_0[\mbox{M}]/k_{\inf})]^2)^{-1}} \\
k_0 & = A_0 e^{\left( \frac{C_0}{T} \right)} \left( \frac{T}{300} \right)^{B_0} \\
k_{inf} & = A_{inf} e^{\left( \frac{C_{inf}}{T} \right)} \left( \frac{T}{300} \right)^{B_{inf}}
\end{split}
\end{equation}

\noindent where $F_C = 0.6$, $N = 1$, $A_0 = 5.9 \times 10^{-33}$, $B_0 = -1.4$, $C_0 = 0$, $A_{inf} = 1.1 \times 10^{-12}$, $B_{inf} = 1.3$, and $C_{inf} = 0$, and a ternary chemical activation rate constant:

\begin{equation}
\begin{split}
k & = \frac{k_0}{1+k_0[\mbox{M}]/k_{\inf}}F_C^{(1+1/N[log_{10}(k_0[\mbox{M}]/k_{\inf})]^2)^{-1}} \\
k_0 & = A_0 e^{\left( \frac{C_0}{T} \right)} \left( \frac{T}{300} \right)^{B_0} \\
k_{inf} & = A_{inf} e^{\left( \frac{C_{inf}}{T} \right)} \left( \frac{T}{300} \right)^{B_{inf}}
\end{split}
\end{equation}

\noindent where $F_C = 0.6$, $N = 1$, $A_0 = 1.5 \times 10^{-13}$, $B_0 = 0.6$, $C_0 = 0$, $A_{inf} = 2.1 \times 10^{9}$, $B_{inf} = 6.1$, and $C_{inf} = 0$.

%%%%%%%%%%%%%%%%%%%%%%%%%%%%%%%%%%%%%%%%%%%%%%%

\section{usr\_oh\_dms}

\begin{blockcode}[commandchars=\\\{\}]
\color{gray}/cam/src/chemistry/mozart/mo_usrrxt.F90 (1777,1784)
       !-----------------------------------------------------------------------
       !     ... DMS + OH --> 0.75 SO2 + 0.25 MSA
       !-----------------------------------------------------------------------
       if ( usr_oh_dms_ndx > 0 ) then
          o2(:ncol) = invariants(:ncol,k,inv_o2_ndx)
          rxt(:,k,usr_oh_dms_ndx) = 2.000e-10_r8 * exp(5820.0_r8 * tinv(:)) / &
               ((2.000e29_r8 / o2(:)) + exp(6280.0_r8 * tinv(:)))
       endif
\end{blockcode}

This does not appear to fit any existing reaction types.

\vspace{20px}
\textit{\Large Should this be added as a reaction type to Music Box? If so, is there a reference to use for the documentation?}


%%%%%%%%%%%%%%%%%%%%%%%%%%%%%%%%%%%%%%%%%%%%%%%

\section{usr\_COhc\_OH, usr\_COme\_OH and usr\_CO01\_OH--usr\_CO42\_OH}

\begin{blockcode}[commandchars=\\\{\}]
\color{gray}/cam/src/chemistry/mozart/mo_usrrxt.F90 (1777,1784)
!-----------------------------------------------------------------
!      ... CO tags
!-----------------------------------------------------------------
      if( usr_CO_OH_b_ndx > 0 ) then
         if( usr_COhc_OH_ndx > 0 ) then
            rxt(:ncol,:,usr_COhc_OH_ndx) = rxt(:ncol,:,usr_CO_OH_b_ndx)
         end if
         if( usr_COme_OH_ndx > 0 ) then
            rxt(:ncol,:,usr_COme_OH_ndx) = rxt(:ncol,:,usr_CO_OH_b_ndx)
         end if
         if( usr_CO01_OH_ndx > 0 ) then
            rxt(:ncol,:,usr_CO01_OH_ndx) = rxt(:ncol,:,usr_CO_OH_b_ndx)
         end if
         if( usr_CO02_OH_ndx > 0 ) then
            rxt(:ncol,:,usr_CO02_OH_ndx) = rxt(:ncol,:,usr_CO_OH_b_ndx)
         end if

         {\color{black}...}

         if( usr_CO41_OH_ndx > 0 ) then
            rxt(:ncol,:,usr_CO41_OH_ndx) = rxt(:ncol,:,usr_CO_OH_b_ndx)
         end if
         if( usr_CO42_OH_ndx > 0 ) then
            rxt(:ncol,:,usr_CO42_OH_ndx) = rxt(:ncol,:,usr_CO_OH_b_ndx)
         end if
\end{blockcode}

Same as \verb>usr_CO_OH_b> (TS1 section 23).

%%%%%%%%%%%%%%%%%%%%%%%%%%%%%%%%%%%%%%%%%%%%%%%

\end{document}